\documentclass{article}

\usepackage[utf8]{inputenc}
\usepackage[english]{babel}
\usepackage{amssymb}
\usepackage{amsthm}
\usepackage{amsfonts}
\usepackage{amsmath}
\usepackage{etoolbox}
\usepackage{enumerate}
\usepackage{breakcites}
\usepackage{float}
\usepackage{algorithm}
\usepackage{algpseudocode, float}
\usepackage{hyperref}
\usepackage{graphicx}

\theoremstyle{definition}
\newtheorem{definition}{Definition}[section]
\AfterEndEnvironment{definition}{\noindent\ignorespaces}

\newtheorem{conjecture}{Conjecture}[section]
\AfterEndEnvironment{conjecture}{\noindent\ignorespaces}

\newtheorem{theorem}{Theorem}[section]
\AfterEndEnvironment{theorem}{\noindent\ignorespaces}

\newtheorem{proposition}{Proposition}[section]
\AfterEndEnvironment{proposition}{\noindent\ignorespaces}

\setlength{\parskip}{.5em}

\title{$k$-Parent Aliquot Numbers: Counting Pre-Images Under the Sum-of-Proper-Divisors Function} 
\author{Gavin Guinn\\{\small Supervisor: Michael J. Jacobson}}
\date{December 2021}

\begin{document}
\maketitle

\section{Introduction}
\subsection{Sum-of-Proper-Divisors Function}
\subsection{Aliquot Sequences}
\subsection{Non-Aliquots and k-Parent Numbers}

\section{Modeling the Density of k-Parent Numbers}
Consider the following set,
$$N_a = \{2,5,52,88,96,120,124,146,162,188,206,210,216,238,246,248, \dots\}$$
which contains non-aliquot numbers \cite{sloane_2022}, if $n \in N_a$ then $\nexists m \text{ s.t. } s(m) = n$ or $\#s^{-1}(n) = 0$. The name non-aliquot implies that these numbers do not appear in aliquot sequences, however every sequence begins with a member of $N_a$; if $\#s^{-1}(n) > 0$ then there must be some preceding number in the same sequence.

We are interested in the natural density of $N_a$, where
$$d(N_a) = \lim_{n \to \infty}\frac{|N_a \cap[1,n]|}{n},$$
in other words determining the proportion of numbers that are non-aliquot. Using probabilistic methods \cite{pollPom} precisely model the density of non-aliquot numbers.
\begin{conjecture}{\textit{The set of non-aliquot numbers has asymptotic density $\Delta$},}
    $$\Delta = \lim_{y \to \infty}\frac{1}{\log y} \sum_{\substack{a\leq y \\ 2 | a}} \frac{1}{a}\text{e}^{-a/s(a)}.$$
\end{conjecture}

This model is an application of the "balls into bins" statistics problem, estimating the number of empty bins after throwing $m$ balls randomly into $n$ bins;
$$\#[\text{empty bins}] = n(1- \frac{1}{n})^m.$$

\section{Tabulation of k-Parent Numbers}
\subsection{Brute-Forcing Preimages}
Consider the equation $s(n) = m$, where the integer $m$ is the \textit{image} of $n$ under the sum-of-proper-divisors function. Computing $s(n)$ is equivalent to the prime-factorization of $n$, using the method of \cite{compute_sumdiv}. A harder problem is to compute the \textit{preimages} of $m$, all integers $n$ such that $s(n) = m$ expressed as $s^{-1}(m)$. Restricting to even values of $m$ this problem can be brute-forced by computing $s(n)$ for ranges of $n$, a conjecture similar to \cite{sumdiv_range} is useful to select the values of $n$.

\begin{conjecture} \textit{If $2 | m$ and $s(n) = m$ then $n \in \{x| \text{ even } x, 2 \leq x \leq 2m \} \cup \{x^2| \text{ odd } x, 1 \leq x \leq m\}$}

    If $2 | n$ then $s(n) = 1+ 2 + \frac{n}{2} + x $ so $s(n) > \frac{n}{2}$;

    \hspace{\parindent}thus $s(n) > m$ if $n > 2m$.

    If $2 \nmid n$ then $s(n) = m$ if and only if $n$ is an odd square,

    \hspace{\parindent}thus $s(n) = 1 + \sqrt{n} + x$ and $s(n) > \sqrt{n}$. So $s(n) > m$ if $n > m^2$.
\end{conjecture}

\begin{proposition} $s(n)$ is even if and only if

    $n = (2m+1)^2$, or

    $2 | n$ and $(n \neq m^2 \text{ or } n \neq 2m^2)$.
\end{proposition}

This approach is expensive if computing $s^{-1}(m)$ for a single value of $m$, most values will be disregarded when searching for $s(n) = m$. It is more palatable to compute preimages for a range of $m$, as determining $s^{-1}(m)$ requires $\{s^{-1}(x)|\text{ even } x,\, 1 \leq x \leq m\}$ anyway. This computation is useful to count the occurrence and density of $k$-parent numbers, these counts can be compared to the predictions of the k-parent heuristic model to provide evidence for correctness.

\subsection{Pomerance-Yang Algorithm}
We can certainly do better than this brute-force approach for enumerating preimages, a technique of \cite{pomYang} is a useful improvement. Algorithm \ref{alg:pomyang} presents the Pomerance-Yang algorithm modified to determine $\#s^{-1}(n)$ for all integers less than or equal to the bound $x$.

\begin{algorithm}[H]
    \caption{Pomerance-Yang}
    \label{alg:pomyang}
    \begin{algorithmic}[1]
        \Procedure{Enumerate\_kParent}{$x$}
        \State{$f[i] \gets 1,\, \forall i \in [1, x]$}
        \ForAll{odd $m \in [1, x]$}
        \If{$2 | \sigma(m)$}
        \State $t \gets 3\sigma(m) - 2m$
        \While{$t \leq x$}
        \State{$f[t] \gets f[t] + 1$}
        \State{$t \gets 2t + \sigma(m)$}
        \EndWhile
        \EndIf
        \If{$\sigma(m) = m+1$} \Comment{if $m$ is prime}
        \State{$f[m+1] \gets f[m+1] + 1$}
        \EndIf
        \EndFor
        \ForAll{odd composite $m \in [1, x^{2/3})$}
        \If{$s(m^2) \leq x$}
        \State{$f[s(m^2)] \gets f[s(m^2)] + 1$}
        \EndIf
        \EndFor
        \State{\textbf{return} $f$}
        \EndProcedure
    \end{algorithmic}
\end{algorithm}

Algorithm \ref{alg:pomyang} was computed on a large scale by \cite{chum_guy_jacobson_mosunov_2018} who used the result to compute the geometric mean of $s(n) / n$ weighted by $\#s^{-1}(n)$. Extending this work to compute the density of $k$-parent numbers I have re-implemented the algorithm and can report a substantial performance improvement.


\bibliographystyle{apalike}
\bibliography{kparent_final_report}
\end{document}
