\textbf{1. Introduction}
                \begin{itemize}
    \item Devitt found the average order of $s(n)/n = 5\pi^2/24 - 1$ over even $n$, why is this result have a exact expression.
    \item Why is the geometric mean rather than the arithmetic mean relevant.
    \item Bosma / Kane calculated that the arithmetic mean of $s(n) / n$ for even $n$ is like 0.967, pomerance continued this computation to more accuracy $\lambda = \log \mu$
    \item pomerance proved $\mu = s_2(n) / s(n)$, does this suggest that $s(n)$ maps to a set that is a exact statistical sample of the even $n$ in respect to the abundance index. This really confuses me as $s(n)$ can map to a odd number. Maybe take a look into this
    \item "The range of s(n) varies", Ima interpret this as the images of $s(n)$ varies. This seems odd to me as isnt the image simply defined as the set of onto which $s(n)$ maps? How could such a thing `vary`, I think it means that there exists $n$ in the domain that doesn't exist in the image of $s(n)$ ie. non-aliquots
    \item the authors are interested in k-parent aliquots as a weight for the calculation of average $s(n) / n$
    \item Pomerance and Yang 2014 suggest that upto a 1/3 of even numbers are not in the range $s(n)$ isnt this like double the tabulated amount of non-aliquots. After taking a look into the paper I genuinely have no idea what this is referring to as Pomerance and Yang report computed densities of like .16
    \item Erdos investigated whether number in the range of $s(n)$ tend to be abundant or deficient. He showed that there are an infinity of abundant numbers not in the range of $s(n)$. It is an open question whether there are an infinite number of defcient numbers out of the range of $s(n)$
    \item Erdos also proved that there are abitrarily long increasing aliquot sequences, ie. for any length k you can choose an m such that the sequence has k steps.
    \item pomerance showed $\log \mu = -0.34$ if for $s(n)/n$ $n$ is restricted to terms that are even and square free. Square free numbers are not divisable by any perfect square, ie. a number where all the prime factors are unique
    \item pomerance showed $\log \mu = -0.24$ if for $s(n)/n$ $n$ is restricted to terms that are conguent to $2 \mod 4$ 
    \item pomerance showed $\log \mu = 0.17$ if for $s(n)/n$ $n$ is restricted to terms that are divisable by 4
    \item I should actually take the time to understand the guide and driver thing
    \item Is this saying that guides and drivers on average tend to force abundance???
    \item A topic of study is where the abundance index of number in the image of s(n) differs from the abundance index of all evens, if numbers in the range of $s(n)$ tend to be abundant then aliquot sequences would be likely unbounded
\end{itemize}

\section{What Drives an aliquot sequence} % figure out some better way to title each section


\textbf{1. Introduction}
\begin{itemize}
\item At $s_8(276)$ terms start being divisible by $2^27 = 28$, since any multiple of 28 is abundant the terms increase monotonically when they have this factor.
\item the 276 sequence loses its driver 28 on the 170th iteration
\item for a prime divisor $p$ of $n$ will appear in $s(n)$ iff $p | \sigma(n)$. What does it mean for $p$ to `appear` in $s(n)$? Does it mean that it is an element of the sum?
\item Arguing in favour of catalan-dickson if most even numbers tend to be abundant then most large even sequences will diverge. This only holds if other effects are minor compared to parity
\item If even values tend to be abundant (let the abundancy index be $\alpha$) we expect the next $r$ term in a sequence to be $n\alpha^r$ where $n$ is the starting value 
\item the probability that any value in a sequence is a square is $c_1 / \sqrt{n\alpha^r}$ (idk what $c_1$ is) the probability that any future term is a square is $c_2 / \sqrt{n}$. Thus the chance of a parity flip goes to zero as n goes to infinity
\item I dont get the proof but they show that a driver can not be expected to persist indefinatly on iterations of $s(n)$
\end{itemize}